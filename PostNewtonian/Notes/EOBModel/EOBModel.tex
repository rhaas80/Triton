\documentclass[twoside, aps, prd, letterpaper, noshowpacs, %
amsmath, amssymb, amsfonts, nofootinbib, floatfix, notitlepage]%
{revtex4-1}
% \documentclass[lengthcheck, aps, prd, letterpaper, noshowpacs, %
% amsmath, amssymb, amsfonts, nofootinbib, floatfix]{revtex4-1}
%\pdfoutput=1

%% A few basic packages
\usepackage{graphicx} %
\usepackage{color} %
\usepackage{bm} %


%% AMS packages for math
\usepackage{amsthm} %
% \usepackage{amssymb} %
% \usepackage{amsmath} %
\usepackage[all]{onlyamsmath} % Error on bad non-ams math


%% Handy packages to have around
%\usepackage{mathrsfs}
%\usepackage{mathabx}
\usepackage{braket}
\usepackage{accents} %
%\usepackage{units} % Makes \units[1.23]{kg} do the right thing
%\usepackage{rotating} % 'sidewaystable' support
\usepackage{xspace} % Adds a space after a text macro
% \usepackage[all]{xy} %% This is the best package for making
%                      %% commutative diagrams


%%% A few selections from txfonts, without using all txfonts
\DeclareSymbolFont{symbolsC}{U}{txsyc}{m}{n}
%\DeclareMathSymbol{\lambdabar}{\mathord}{symbolsC}{111}
\DeclareMathSymbol{\coloneqq}{\mathrel}{symbolsC}{066}



%%% Local Variables: 
%%% mode: latex
%%% TeX-master: "Conventions"
%%% End: 


%%%%%%%%%%%%%%%%%%%%%%%%%%%%%%%%%%%%%%%%%%%%%%%%%%%%%%%%%%%%%%%%%%%%%%
%% This is all for the hyperref package, and nice colors
\RequirePackage{ifpdf}
\ifpdf
\usepackage[backref, colorlinks, hyperindex, plainpages=false, %
bookmarksnumbered, pdfpagemode=UseNone, hyperfigures=true]{hyperref}
\usepackage{thumbpdf}
\renewcommand*{\backref}[1]{} 
\renewcommand*{\backrefalt}[4]{% 
  \ifcase #1 { } \or (Cited in Sec.~#2.) \else %
  (Cited in Secs.~#2.) \fi %
}
\else
\usepackage[colorlinks, pdfborder={0 0 0}, hyperindex,%
plainpages=false]{hyperref}
\fi
\definecolor{ChapterColor}{rgb}{0.510, 0.118, 0.000}
\definecolor{LinkColor}   {rgb}{0.741, 0.173, 0.000}
\definecolor{CiteColor}   {rgb}{0.000, 0.502, 0.118}
\definecolor{UrlColor}    {rgb}{0.000, 0.212, 0.800}
\hypersetup{linkcolor=LinkColor}
\hypersetup{citecolor=CiteColor}
\hypersetup{urlcolor=UrlColor}
%\hypersetup{pdflang={English}}
%%%%%%%%%%%%%%%%%%%%%%%%%%%%%%%%%%%%%%%%%%%%%%%%%%%%%%%%%%%%%%%%%%%%%%


%%% Local Variables: 
%%% mode: latex
%%% TeX-master: "Conventions"
%%% End: 

%%%%%%%%%%%%%%%%
%% Math forms %%
%%%%%%%%%%%%%%%%
\renewcommand{\d}{\ensuremath{\mathrm{d}}}
\newcommand{\e}{\ensuremath{\mathrm{e}}}
\renewcommand{\i}{\ensuremath{\mathrm{i}}}
\newcommand{\scri}{\ensuremath{\mathcal{I}}}
\newcommand{\Mirr}{\ensuremath{M_{\text{irr}}}}
\newcommand{\Eadm}{\ensuremath{M_{\text{ADM}}}}
\newcommand{\Jadm}{\ensuremath{J_{\text{ADM}}}}
\newcommand{\ADMMass}{\Eadm}
\newcommand{\IrrMass}{\Mirr}
\newcommand{\hyb}{\text{hyb}\xspace}
\newcommand{\NR}{\text{NR}\xspace}
\newcommand{\PN}{\text{PN}\xspace} \newcommand{\pN}{\PN}
\newcommand{\fc}{\ensuremath{f_{c}}}
\newcommand{\deff}{\ensuremath{D_{\text{eff}}}} % effective distance
\newcommand{\ta}{\ensuremath{t_{\text{a}}}}
\newcommand{\phia}{\ensuremath{\phi_{\text{a}}}}

%% Units
\renewcommand{\c}{\mathrm{c}}
\newcommand{\G}{\mathrm{G}}
\newcommand{\MSun}{\ensuremath{M_\odot}\xspace}
\newcommand{\Sun}{\MSun}

%% abs, norm, ceil, floor, avg
\newcommand{\abs} [1]{\left\lvert{#1}\right\rvert}
\newcommand{\norm}[1]{\left\lVert{#1}\right\rVert}
\newcommand{\ceil}[1]{\left\lceil{#1}\right\rceil}
\newcommand{\floor}[1]{\left\lfloor{#1}\right\rfloor}
\newcommand{\avg}[1]{\left\langle{#1}\right\rangle}

%% Frequencies and discretization
\newcommand{\fsamp}{\ensuremath{f_{\mathrm{s}}}}
\newcommand{\fNy}{\ensuremath{f_{\mathrm{Ny}}}}
\newcommand{\discretize}{\rightsquigarrow}
\DeclareMathOperator{\fft}{fft}
\DeclareMathOperator{\ifft}{ifft}


%%%%%%%%%%%%%%%%
%% Text forms %%
%%%%%%%%%%%%%%%%
\newcommand{\foreign}[1]{\textrm{#1}}
\newcommand{\etal}{\foreign{et~al.}\xspace}
\newcommand{\eg}{\foreign{e.g.}\xspace}
\newcommand{\ie}{\foreign{i.e.}\xspace}
\newcommand{\etc}{\foreign{etc.}\xspace}

\definecolor{NoteColor}{rgb}{0.900, 0.218, 0.000}
\newcommand{\Note}[1]{\textcolor{NoteColor}{[#1]}\index{Note}}
\definecolor{NewColor}{rgb}{0,.55,0}
\newcommand{\New}[1]{\textcolor{NewColor}{#1}\index{New}}

\newcommand{\software}[1]{\textsc{#1}}
\newcommand{\command}[1]{\texttt{#1}}


%%%%%%%%%%%%%%
%% Commands %%
%%%%%%%%%%%%%%
\newcommand{\CapName}[1]{\textbf{#1}.}

%% The following command allows you to enter \ShowDimensions anywhere
%% in a document, and -- at processing -- see all the dimensions (I
%% could think of) in the terminal.
\makeatletter
\newcommand{\ShowDimensions}{%
  \typeout{The font encoding is \f@encoding}        %
  \typeout{The font family is \f@family}            %
  \typeout{The font series is \f@series}            %
  \typeout{The font shape is \f@shape}              %
  \typeout{The font size is \f@size}                %
  \typeout{The baselineskip is \f@baselineskip}     %
  \typeout{The math font size is \tf@size}          %
  \typeout{The math script size is \sf@size}        %
  \typeout{The math scriptscript size is \ssf@size} %
  \typeout{The linewidth is \the\linewidth}         %
}
\makeatother


%%%%%%%%%%%%%%%%%%%%%%%%%%%%%%%%%%%%%%%%%%%%%%%%%%%%%%%%%%%%%%%%%%
%% My extension of the braket package for inner products        %%
%% This allows code like $\InnerProduct{s|h}$, which works much %%
%% like $\Braket{x|y}$.                                         %%
%%%%%%%%%%%%%%%%%%%%%%%%%%%%%%%%%%%%%%%%%%%%%%%%%%%%%%%%%%%%%%%%%%
\makeatletter
\let\protect\relax
{\catcode`\|=\active
  \xdef\InnerProduct{\protect\expandafter\noexpand\csname InnerProduct \endcsname}
  \expandafter\gdef\csname InnerProduct \endcsname#1{%
    \begingroup
    \ifx\SavedDoubleVert\relax
    \let\SavedDoubleVert\|\let\|\IpDoubleVert
    \fi
    \mathcode`\|32768\let|\IPVert
    \left({#1}\right)
    \endgroup
  }
}
\def\IPVert{\@ifnextchar|{\|\@gobble}% turn || into \|
     {\egroup\,\mid@vertical\,\bgroup}}
\def\IPDoubleVert{\egroup\,\mid@dblvertical\,\bgroup}
\let\SavedDoubleVert\relax
\def\midvert{\egroup\mid\bgroup}
\def\SetVert{\@ifnextchar|{\|\@gobble}% turn || into \|
    {\egroup\;\mid@vertical\;\bgroup}}
\def\SetDoubleVert{\egroup\;\mid@dblvertical\;\bgroup}
\def\mid@vertical{\mskip1mu\vrule\mskip1mu}
\def\mid@dblvertical{\mskip1mu\vrule\mskip2.5mu\vrule\mskip1mu}
\makeatother
\newcommand{\Overlap}{\Braket}


%%% Local Variables: 
%%% mode: latex
%%% TeX-master: "Conventions"
%%% End: 


\newcommand{\Caltech}{\affiliation{Theoretical Astrophysics 350-17,
    California Institute of Technology, Pasadena, CA 91125}}
\newcommand{\Cornell}{\affiliation{Center for Radiophysics and Space
    Research, Cornell University, Ithaca, New York, 14853}}


%%% Local Variables: 
%%% mode: latex
%%% TeX-master: "Conventions"
%%% End: 


\allowdisplaybreaks[4]

\usepackage[absolute]{textpos}
\setlength{\TPHorizModule}{1.35in}
\setlength{\TPVertModule}{\TPHorizModule}

\newcommand{\drstardr}{\frac{d\rstar}{dr}}

%%%%%%%%%%%%%%%%

\begin{document}

%%%%%%%%%%%%%%%%


% \graphicspath{%
%   {Plots/}%
% }

\title[EOB for dummies] {EOB for dummies}

\surnames{Boyle} 

\author{Michael Boyle} \Cornell

\begin{abstract}
  Here we explicitly give all necessary components of the EOB model
  collected in one place, to ease implementation and ensure clarity.
  We first give the basic equations to compute the inspiral phasing.
  We then describe the waveform amplitude, which is required when
  evolving the phase.  We describe the ringdown carefully and exhibit
  the method for joining the inspiral and ringdown in detail.
  Finally, we also give a suggested implementation for the LIGO
  Algorithm Library.  For simplicity, we treat all quantities as
  dimensionless.
\end{abstract}

\date{\today}

% \pacs{%
%   04.30.-w, % Gravitational waves
%   04.80.Nn, % Gravitational wave detectors and experiments
%   04.25.dg % NR studies of black holes and black-hole binaries
% }

% 04.25.-g, % Approximation methods; equations of motion
% 04.25.D-, % Numerical relativity
% 04.25.dc, % NR studies of crit. behavior, sing.'s, cosmic censorsh.
% 04.25.dg, % NR studies of black holes and black-hole binaries
% 04.25.dk, % NR studies of other relativistic binaries
% 04.25.Nx, % PN approximation; perturbation theory; etc.
% 04.30.-w, % Gravitational waves
% 04.30.Db, % Wave generation and sources
% 04.30.Nk, % Wave propagation and interactions
% 04.30.Tv, % Gravitational-wave astrophysics
% 04.80.Nn, % Gravitational wave detectors and experiments

\maketitle


%%%%%%%%%%%%%%%%%%%%%%%%%%%%%%%%%%%%%%%%%%%%%%%%%%%%%%%%%%%%%%%%%%%%%%
%%%%%%%%%%%%%%%%%%%%%%%%%%%%%%%%%%%%%%%%%%%%%%%%%%%%%%%%%%%%%%%%%%%%%%

\begin{textblock}{1}(-.49,-.05)
  \rotatebox{-36}{\includegraphics[height=1.35in]{Dummie}}
\end{textblock}

\section{Inspiral orbital phase}
\label{sec:InspiralOrbitalPhase}
The phasing of the EOB system during inspiral consists of four basic
ingredients.  These four ingredients are changed from one paper to the
next, but are generally combined in a standard way.  The ingredients
are:
\begin{enumerate}
 \item the effective metric;
 \item the flux of energy $\Flux$ carried away by gravitational waves;
 \item the relation of $\Flux$ to the flux of angular momentum
  $\EOBTorque$ being carried away; and
 \item the Hamiltonian $H$.
\end{enumerate}
In the non-precessing case, the system is described by coordinates $r$
and $\OrbitalPhase$ (describing the position of the reduced mass
moving in the effective background), which are obtained by evolving
the Hamilton equations:\footnote{The form of these equations is fairly
  standard, except for the $\EOBTorque$ term in the equation for
  $d\prstar/dt$, which is a small correction throughout most of the
  inspiral.}
\begin{equation}
  \label{eq:HamiltonEquations}
  \begin{aligned}
    \frac{dr}{dt} &= \frac{dr}{d\rstar}\, \frac{\partial H}
    {\partial \prstar}~; \qquad \qquad & \frac{d \prstar}{d t} &=
    -\frac{dr}{d\rstar}\, \frac{\partial H} {\partial r}
    -\EOBTorque\, \frac{\prstar}
    {\pPhi}~;
    \\
    \frac{d \OrbitalPhase}{d t} &= \frac{\partial H} {\partial
      \pPhi}~; & \frac{d \pPhi}{d t} &= -\EOBTorque~.
  \end{aligned}
\end{equation}
The tortoise coordinate $\rstar$ is introduced in an attempt to keep
the equations more regular near the ``horizon'', and can be computed
from the metric components given below.  The remainder of this section
explicitly presents the ingredients necessary to construct and
integrate these equations.

\subsection{Effective metric}
The basic EOB model is based on a static, spherically symmetric
effective metric:\footnote{The metric components $A(r)$ and $D(r)$ can
  be computed as power series in $1/r$, using known PN formulas.
  These power series are written in the form $A(r) =
  \sum_{j=0}^{k+1}\, \frac{a_{j}}{r^{j}}$, $D(r) = \sum_{j=0}^{k}\,
  \frac{d_{j}}{r^{j}}$.  The series are known up to $k=3$---that is,
  $a_{4}$ and $d_{3}$ are known.  Beyond that, the parameters are
  ``calibrated'' to match numerical simulations.  Logarithmic terms
  may appear at $k=4$, but typically are not modeled.  These are the
  series that are expanded in \Pade form in
  Eqs.~\eqref{MetricCoefficientA} and~\eqref{eq:MetricCoefficientD}.}
\begin{equation}
  \label{eq:EOBmetric}
  ds_\text{eff}^2 = -A(r)\,dt^2 + \frac{D(r)}{A(r)}\,dr^2 +
  r^2\,\left(d\Theta^2+\sin^2\Theta\,d\OrbitalPhase^2\right)~,
\end{equation}
The equations of motion describe a particle moving in this effective
background.  We begin by expanding the metric components in
Eq.~\eqref{eq:EOBmetric} as \Pade approximants.  First, we use the
coefficient $A(r)$ in \Pade form:
\begin{gather}
  \label{MetricCoefficientA}
  A(r) = A_5^1(r) = \frac{\mathrm{Num}(A_5^1)}{\mathrm{Den}(A_5^1)}~,
  \\
  \intertext{where}
  \begin{split}
    \mathrm{Num}(A_5^1) = r^4\,\left[-64 + 12\,a_4+4\,a_5+a_6+64 \nu-4
      \nu ^2 \right] + r^5\,\left[32-4\,a_4-a_5-24 \nu \right]
  \end{split}
  \\ \intertext{and}
  \begin{split}
    \mathrm{Den}(A_5^1) &=
    4\,a_4^2 + 4\,a_4\,a_5 + a_5^2 - a_4\,a_6 + 16\,a_6 +
    (32\,a_4 + 16\,a_5 - 8\,a_6)\,\nu + 4\,a_4\,\nu^2 + 32\,\nu ^3 \\
    &\quad + r\,\left[ 4\,a_4^2 + a_4\,a_5 + 16\,a_5 + 8\,a_6 +
      (32\,a_4-2\,a_6)\,\nu + 32\,\nu^2 + 8\,\nu^3 \right]  \\
    &\quad + r^2\,\left[16\,a_4 + 8\,a_5 + 4\,a_6 +
      (8\,a_4+2\,a_5)\,\nu + 32\,\nu^2\right] \\
    &\quad + r^3\,\left[8\,a_4+4\,a_5+2\,a_6+32\,\nu-8\,\nu ^2\right]
    \\
    &\quad + r^4\,\left[4\,a_4+2\,a_5+a_6+16\,\nu-4\,\nu ^2\right] \\
    &\quad + r^5\,\left[32-4\,a_4-a_5+24\,\nu\right]~.
  \end{split}
\end{gather}
The coefficient $a_{4}$ is known from post-Newtonian calculations, and
$a_{5}$ and $a_{6}$ are fit in Ref.~\cite{PanEtAl:2011}, obtaining
\begin{subequations}
  \label{eq:a4a5a6values}
  \begin{gather}
    a_{4} = \left( \frac{94}{3} - \frac{41}{32}\, \pi^{2} \right)\,
    \nu~, \\
    a_{5} = (-5.828\,\nu - 143.5\,\nu^{2} + 447\,\nu^3)~, \\
    a_{6} = 184\, \nu~.
  \end{gather}
\end{subequations}
The other metric coefficient $D(r)$ is also expanded in \Pade form:
\begin{equation}
  \label{eq:MetricCoefficientD}
  D(r) = D_{3}^{0}(r) = \frac{r^{3}} {(52\,\nu - 6\,\nu^{2}) + 6\,
    \nu\, r + r^{3}}~.
\end{equation}
Using the expressions of Eqs.~\eqref{MetricCoefficientA}
and~\eqref{eq:MetricCoefficientD}, we can also calculate the relation
between the standard and tortoise radii needed for the Hamilton
equations:
\begin{equation}
  \label{eq:TortoiseRadius}
  \frac{dr}{d\rstar} = \frac{A(r)}{\sqrt{D(r)}}~.
\end{equation}


\subsection{Energy and angular momentum fluxes}
Standard Hamiltonian systems are conservative---no energy can be lost.
The EOB model incorporates the loss of energy in the form of
gravitational waves by including terms expressing the forces they
exert on the system.  In the Hamilton equations,
Eq.~\eqref{eq:HamiltonEquations}, those are the terms proportional to
$-\EOBTorque$, which is the torque acting on the system due to the
angular momentum being carried away by the gravitational waves.  This
can be related to the flux of energy $\Flux$ carried off by the
gravitational waves as\footnote{Note that this expression is not
  unique in the literature.  Other references have introduced more
  complicated expressions similar to this, meant to incorporate
  non-Keplerian corrections~\cite{DamourEtAl:1998}.  Here,
  non-Keplerian corrections are included to the waveform instead.}
\begin{equation}
  \label{eq:TorqueAndFlux}
  \EOBTorque = \frac{1}{\nu\, \OrbitalFreq}\, \Flux~,
\end{equation}
where $\OrbitalFreq \define d\OrbitalPhase / dt$.  The energy flux is
given by\footnote{Again, this form of the flux is not the only choice
  used in the EOB literature.  Many different approximations have been
  used in its place.  In particular, simpler flux approximants are
  used when computation time is critical, as in LAL.}
\begin{subequations}
  \label{eq:Flux}
  \begin{align}
  \label{eq:FluxExact}
    \Flux(r, \OrbitalFreq, \prstar, \pPhi) &= \frac{1}{16\, \pi}\,
    \sum_{\ell=2}^{\infty}\, \sum_{m=-\ell}^{\ell}\,
    \abs{\frac{\mathcal{R}}{M}\, \dot{h}_{\ellm} (r, \OrbitalFreq,
      \prstar, \pPhi)}^{2}~,
    \\ \label{eq:FluxApprox}
    & \approx \frac{\OrbitalFreq^{2}} {8\, \pi}\, \sum_{\ell}\,
    \sum_{m\geq 0}\, m^{2}\, \abs{ \frac{\mathcal{R}}{M}\, h_{\ellm}
      (r, \OrbitalFreq, \prstar, \pPhi)}^{2}~.
  \end{align}
\end{subequations}
The approximation in the second line is valid in the slow-motion
limit, where the change in the waveform is dominated by the frequency
rather than the changing amplitude.  Also, if the modes are explicitly
summed, computational efficiency dictates that only the modes with the
largest expected amplitude be included in the sum.  The quantity
$h_{\ellm}$ is the waveform amplitude, given by
Eq.~\eqref{eq:WaveformAmplitude} below, and $\mathcal{R}/M$ simply
factors out the dependence on $\mathcal{R}$ and $M$ (so that the final
equations do not involve those parameters).


\subsection{Hamiltonian}
As described above, the binary is evolved as a standard Hamiltonian
system, where the Hamiltonian function is\footnote{This form of the
  Hamiltonian is fairly standard for nonspinning systems throughout
  the literature.  The $\prstar^{4}$ term should really have a factor
  of $(d\rstar/dr)^{4}$, but this is dropped because it is formally of
  a higher PN order.  (Although its magnitude would be very important
  in that term by the time $r \approx 8$, for example.)}
\begin{equation}
  \label{eq:Hamiltonian}
  H = \frac{1}{\nu}\, \sqrt{1 + 2\, \nu\, \left(
      \sqrt{\prstar^{2} + A(r) \left[ 1 + \frac{\pPhi^{2}}{r^{2}} +
          (8\, \nu - 6\, \nu^{2})\, \frac{\prstar^{4}} {r^{2}}
        \right]} - 1 \right)} - \frac{1}{\nu}~.
\end{equation}

\section{Factorized inspiral waveform}
\label{sec:InspiralWaveform}

To define the waveform amplitude, we need to define a number of
auxiliary functions.  First, three velocity parameters:
\begin{gather}
  \label{eq:vOrbitalFreq}
  v_{\OrbitalFreq} = \OrbitalFreq^{1/3}~,
  \\ \label{eq:vPhi}
  \vPhi = \OrbitalFreq\, \left\{ \frac{\partial H(r, \prstar=0,
      \pPhi)} {\partial \pPhi} \right\}^{-2/3} = \OrbitalFreq\,
  \left\{ r\, \frac{2 + 4\, \nu\, \left[ \sqrt{A(r)\, (1 +
          \pPhi^{2}/r^{2})} - 1 \right]} {dA(r)/dr} \right\}^{1/3}~,
  \\ \label{eq:Vlm}
  V_{\ell,m} =
  \begin{cases}
    \OrbitalFreq\, \vPhi^{\ell+\epsilon-3} & \text{$(\ell, m) = (2,1)$
      or $(4,4)$;} \\
    \vPhi^{\ell+\epsilon} & \text{otherwise.}
  \end{cases}
\end{gather}
In Eq.~\eqref{eq:vPhi}, $A(r)$ and its derivative are computed from
the \Pade form given in Eq.~\eqref{MetricCoefficientA}.  In
Eq.~\eqref{eq:Vlm}, the parity $\epsilon$ is defined by
\begin{equation}
  \label{eq:DefineParity}
  \epsilon =
  \begin{cases}
    0 & \text{$\ell + m$ even;} \\
    1 & \text{$\ell + m$ odd.}
  \end{cases}
\end{equation}
Next, we introduce a couple convenient normalization constants:
\begin{gather}
  \label{eq:DefineNormalizationConstantn}
  n_{\ellm} =
  \begin{cases}
    (i\, m)^\ell\, \frac{8\, \pi}{(2 \ell+1)!!}\, \sqrt{\frac{(\ell+1)
        (\ell+2)}{(\ell-1) \ell}} & \text{$\ell + m$ even;} \\
    -(i\, m)^\ell\, \frac{16\, \pi\, i}{(2\, \ell+1)!!}\,
    \sqrt{\frac{(\ell+2) (2\, \ell+1) \left(\ell^2-m^2\right)}
      {(\ell-1) \ell (\ell+1) (2\, \ell-1)}} & \text{$\ell + m$ odd;}
  \end{cases}
  \\
  \label{eq:DefineNormalizationConstantc}
  c_{k} = \left( \frac{1 + \sqrt{1-4\, \nu }}{2}
  \right)^{k -1} + (-1)^{k}
  \left(\frac{1-\sqrt{1-4\, \nu}}{2}\right)^{k-1}~.
\end{gather}

The full waveform is refactorized as five terms multiplying the
Newtonian waveform: an effective source term
$S_{\text{eff}}^{(\epsilon)}$; a tail term $T_{\ellm}$; a phase term
$\delta_{\ellm}$; an amplitude-correction term $\rho_{\ellm}$; and a
non-quasicircular term $\NQC_{\ellm}$.  Except for calibrations listed
below, the phase terms $\delta_{\ellm}$ can be read from Eqs.~(26)
and~(E1), and the amplitude terms $\rho_{\ellm}$ from Eqs.~(29)
and~(D1) of Ref.~\cite{PanEtAl:2010b}.\footnote{In that reference, $q$
  denotes the spin parameter, which we set to zero for this paper, $v$
  should be replaced by $\vOmega$, and the superscript $L$ on some
  expressions is irrelevant for our purposes.  (In particular, we use
  the expressions with superscript $L$, and ignore alternative
  expressions appearing later with superscript $H$.)}  The nonzero
calibrations are
\begin{gather}
  \label{eq:CalibratedDeltaAndRhoTerms}
  \tilde{\delta}_{2,1} = 30\, \nu\, \vOmega^{7/2}~, \\
  \tilde{\delta}_{3,3} = -10\, \nu\, \vOmega^{7/2}~, \\
  \tilde{\delta}_{4,4} = -70\, \nu\, \vOmega^{5/2}~, \\
  \tilde{\delta}_{5,5} = 40\, \nu\, \vOmega^{5/2}~, \\
  \tilde{\rho}_{2,1} = -5\, \nu\, \vOmega^{3}~, \\
  \tilde{\rho}_{3,3} = -20\, \nu\, \vOmega^{3}~, \\
  \tilde{\rho}_{4,4} = -14\, \nu\, \vOmega^{3}~, \\ %\intertext{and}
  \tilde{\rho}_{4,4} = 4\, \nu\, \vOmega^{3}~.
\end{gather}
The remaining factorization terms are defined as follows.
\begin{gather}
  \label{eq:DefineSourceTerm}
  S_{\text{eff}}^{(\epsilon)} =
  \begin{cases}
    \sqrt{\prstar^{2} + A(r) \left[ 1 + \frac{\pPhi^{2}}{r^{2}} + (8\,
        \nu - 6\, \nu^{2})\, \frac{\prstar^{4}} {r^{2}} \right]} &
    \epsilon=0~;
    \\
    \pPhi\, \vOmega & \epsilon=1~;
  \end{cases}
  \\
  \label{eq:DefineTailTerm}
  T_{\ellm} = \frac{\Gamma(\ell + 1 - 2\, i\, m\, \OrbitalFreq\, H)}
  {\Gamma(\ell + 1)}\, \exp \left\{ m\, \OrbitalFreq\, H\, \left[\pi +
      2\, i\, \log(2\, m\, \OrbitalFreq / \sqrt{e}) \right] \right\}~;
  \\
  \label{eq:DefineNQCTerm}
  \NQC_{\ellm} = \left( 1 + a_{1}^{h_{\ellm}}\, \frac{\prstar^{2}}
    {r^{2}\, \OrbitalFreq^{2}} + a_{2}^{h_{\ellm}}\,
    \frac{\prstar^{2}} {r^{3}\, \OrbitalFreq^{2}} +
    a_{3}^{h_{\ellm}}\, \frac{\prstar^{2}} {r^{7/2}\,
      \OrbitalFreq^{2}} \right) \times \exp \left[ i\, \left(
      b_{1}^{h_{\ellm}}\, \frac{\prstar} {r \OrbitalFreq} +
      b_{2}^{h_{\ellm}}\, \frac{\prstar^{3}} {r\, \OrbitalFreq}
    \right) \right]~.
\end{gather}
Here, $\Gamma$ is the usual Gamma function.  The parameters
$a_{i}^{h_{\ellm}}$, and $b_{i}^{h_{\ellm}}$ are given by the fit in
Table~III of Ref.~\cite{PanEtAl:2011}.

Finally, we can combine these elements to write the waveform
amplitude:
\begin{equation}
  \label{eq:WaveformAmplitude}
  \frac{\mathcal{R}}{M}\, h_{\ellm} = \nu\, n_{\ellm}\, c_{\ell +
    \epsilon}\, V_{\ell,m}\, Y^{\ell-\epsilon, -m}
  \left(\frac{\pi}{2}, \OrbitalPhase \right)\,
  S_{\text{eff}}^{(\epsilon)}\, T_{\ellm}\, e^{i\, (\delta_{\ellm} +
    \tilde{\delta}_{\ellm})}\, (\rho_{\ellm} +
  \tilde{\rho}_{\ellm})^{\ell}\, \NQC_{\ellm}~,
\end{equation}
where $Y$ is the standard (spin weight 0) spherical harmonic.  This
expression is used both to obtain the orbital phase in
Eq.~\eqref{eq:Flux}, and to obtain the waveform itself.

\section{Initial data}
\label{sec:EOBInitialData}
\Note{Stuff goes here.}


\section{Attaching the ringdown}
\label{sec:AttachingTheRingdown}
\Note{Stuff goes here.}


\section{Including spin}
\label{sec:IncludingSpin}
\begin{align}
  \label{eq:SpinRelations}
  \Spin &\define \Spin_{1} + \Spin_{2} & \Chi &\define
  \frac{\Spin}{M^{2}} = (1 - 2\,\nu)\, \ChiS
  + \delta\, \ChiA \\
  \SpinStar &\define \frac{M_{2}}{M_{1}}\, \Spin_{1} +
  \frac{M_{1}}{M_{2}}\, \Spin_{2} & \ChiStar &\define
  \frac{\SpinStar}{M^{2}} = 2\, \nu\, \ChiS \\
  \SpinKerr &= \Spin + \SpinStar & \ChiKerr &\define
  \frac{\SpinKerr}{M^{2}} = \Chi + \ChiStar
\end{align}

\begin{equation}
  \label{eq:iota}
  \iota \define r^{2} + \ChiKerr^{2}
\end{equation}

\begin{equation}
  \label{eq:kappa}
  \kappa \define \iota - \Delta_{t}
\end{equation}

\begin{equation}
  \label{eq:lambda}
  \lambda \define \iota^{2} - \ChiKerr^{2}\,
  \Delta_{t}\, \sin^{2} \theta
\end{equation}

\begin{equation}
  \label{eq:rho}
  \rho \define \sqrt{r^{2} + \ChiKerr^{2}\, \cos^{2} \theta}
\end{equation}

\begin{equation}
  \label{eq:alpha}
  \alpha \define \rho\, \sqrt{ \frac{\Delta_{t}} {\lambda} }
\end{equation}

\begin{equation}
  \label{eq:beta}
  \Vector{\beta} \define \frac{\kappa} {\lambda}\, (\ChiKerr \times \Vector{r})
\end{equation}

\begin{equation}
  \label{eq:tau}
  \Vector{\tau} \define \frac{\kappa} {\lambda}\, (\Vector{r} \times
  \Vector{p})
\end{equation}

\begin{equation}
  \label{eq:gamma}
  \gamma^{ij} = \frac{1}{\rho^{2}}\, \left[ \Delta_{r}\, n^{i}\, n^{j}
    - r^{2}\, (\delta^{ij} - n^{i}\, n^{j}) \right] - \frac{1}
  {\rho^{2}\, \Delta_{t}}\, (\ChiKerr \times \Vector{r})^{i}\,
  (\ChiKerr \times \Vector{r})^{j} - \frac{\beta^{i}\, \beta^{j}}
  {\alpha^{2}}
\end{equation}

\begin{equation}
  \label{eq:sigma}
  \begin{split}
    \Vector{\sigma} &\define \frac{\pPhi^2} {r^2}\, \left[\frac{1}{2}
      \Chi \left(a + \frac{3 \nu} {8} \right) + \frac{1}{2} \ChiStar
      \left( b - \frac{\nu}{2} - \frac{5} {8} \right) \right] -
    \pr^2\, \frac{\Delta_{r}} {r^{2}}\, \left[ \Chi \left( a +
        \frac{33\, \nu} {16} \right) + \ChiStar \left( b + \frac{26}
        {16}\, \nu + \frac{5} {16} \right) \right]
    \\
    &\phantom{\define~} - \frac{1} {r}\, \left[\frac{1}{2} \Chi
      (a+\nu) + \frac{1}{8} \ChiStar (4 b + 5 \nu + 2) \right] -
    \frac{\ChiStar} {4}
  \end{split}
\end{equation}

\begin{equation}
  \label{eq:HeffSpin}
  \Heff = \alpha\, \sqrt{1 + \gamma^{ij}\, p_{i}\, p_{j} +
    \frac{(8-6\, \nu)\, \nu\, \pr^{4}} {\rho^{2}}} + \beta^{i}\, p_{i} +
  \tau^{i}\, \sigma_{i} + a_{\text{SS}}\, \nu\, \frac{\ChiKerr \cdot
    \ChiStar} {r^{4}}
\end{equation}

\subsection{Non-precessing systems}
\label{sec:NonprecessingSystems}

\begin{equation}
  \label{eq:HeffSpin}
  \Heff = \sqrt{ \frac{\iota^{2}}{\lambda}\,
    \prstar^{2} + \frac{r^{2}\, \Delta_{t}} {\lambda} \left[ 1 +
      \frac{r^{2}\, \pPhi^{2}} {\lambda} + \frac{(8-6\, \nu)\, \nu\,
        (\prstar\, \drstardr)^{4}} {r^{2}} \right] } + \frac{\kappa\,
    (\chiKerr + \sigma)} {\lambda}\, \pPhi + a_{\text{SS}}\, \nu\,
  \frac{\chiKerr\, \chiStar} {r^{4}}
\end{equation}
[Check that the last $\pPhi$ term shouldn't have another $r$ factor.]

\begin{equation}
  \label{eq:HamiltonianSpin}
  H = \frac{1}{\nu}\, \sqrt{1 + 2\, \nu\, \left(
      \sqrt{\Heff} - 1 \right)} - \frac{1}{\nu}~.
\end{equation}

%%%%%%%%%%%%%%%%%%%%%%%%%%%%%%%%%%%%%%%%%%%%%%%%%%%%%%%%%%%%%%%%%%%%%%
%%%%%%%%%%%%%%%%%%%%%%%%%%%%%%%%%%%%%%%%%%%%%%%%%%%%%%%%%%%%%%%%%%%%%%
% \begin{acknowledgments}
%   It is my pleasure to thank \Note{?}.  This project was supported
%   in part by a grant from the Sherman Fairchild Foundation; by NSF
%   Grants No.\ PHY-0969111 and No.\ PHY-1005426; and by NASA Grant
%   No.\ NNX09AF96G. The numerical computations presented in this
%   paper were performed primarily on the \texttt{Zwicky} cluster
%   hosted at Caltech by the Center for Advanced Computing Research,
%   which was cofunded by the Sherman Fairchild Foundation and the NSF
%   MRI-R\textsuperscript{2} program.
% \end{acknowledgments}

%%%%%%%%%%%%%%%%%%%%%%%%%%%%%%%%%%%%%%%%%%%%%%%%%%%%%%%%%%%%%%%%%%%%%%
%%%%%%%%%%%%%%%%%%%%%%%%%%%%%%%%%%%%%%%%%%%%%%%%%%%%%%%%%%%%%%%%%%%%%%
\appendix* % Use \appendix* if there is just one appendix
% \appendix % Use \appendix if there are multiple appendices

% \section{Approximating gravitational waveforms}



%%%%%%%%%%%%%%%%%%%%%%%%%%%%%%%%%%%%%%%%%%%%%%%%%%%%%%%%%%%%%%%%%%%%%%
%%%%%%%%%%%%%%%%%%%%%%%%%%%%%%%%%%%%%%%%%%%%%%%%%%%%%%%%%%%%%%%%%%%%%%
%\clearpage
\bibliography{Externals/References}


%%%%%%%%%%%%%%

\end{document}

%%%%%%%%%%%%%%


%%% Local Variables: 
%%% mode: latex
%%% TeX-master: t
%%% End: 
