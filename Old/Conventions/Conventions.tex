\documentclass[aps, prd, amsfonts, amssymb, amsmath, %twocolumn, %
%showpacs, showkeys, 
nofootinbib]{revtex4}

%% A few basic packages
\usepackage{graphicx} %
\usepackage{color} %
\usepackage{bm} %


%% AMS packages for math
\usepackage{amsthm} %
% \usepackage{amssymb} %
% \usepackage{amsmath} %
\usepackage[all]{onlyamsmath} % Error on bad non-ams math


%% Handy packages to have around
%\usepackage{mathrsfs}
%\usepackage{mathabx}
\usepackage{braket}
\usepackage{accents} %
%\usepackage{units} % Makes \units[1.23]{kg} do the right thing
%\usepackage{rotating} % 'sidewaystable' support
\usepackage{xspace} % Adds a space after a text macro
% \usepackage[all]{xy} %% This is the best package for making
%                      %% commutative diagrams


%%% A few selections from txfonts, without using all txfonts
\DeclareSymbolFont{symbolsC}{U}{txsyc}{m}{n}
%\DeclareMathSymbol{\lambdabar}{\mathord}{symbolsC}{111}
\DeclareMathSymbol{\coloneqq}{\mathrel}{symbolsC}{066}



%%% Local Variables: 
%%% mode: latex
%%% TeX-master: "Conventions"
%%% End: 


%%%%%%%%%%%%%%%%%%%%%%%%%%%%%%%%%%%%%%%%%%%%%%%%%%%%%%%%%%%%%%%%%%%%%%
%% This is all for the hyperref package, and nice colors
\RequirePackage{ifpdf}
\ifpdf
\usepackage[backref, colorlinks, hyperindex, plainpages=false, %
bookmarksnumbered, pdfpagemode=UseNone, hyperfigures=true]{hyperref}
\usepackage{thumbpdf}
\renewcommand*{\backref}[1]{} 
\renewcommand*{\backrefalt}[4]{% 
  \ifcase #1 { } \or (Cited in Sec.~#2.) \else %
  (Cited in Secs.~#2.) \fi %
}
\else
\usepackage[colorlinks, pdfborder={0 0 0}, hyperindex,%
plainpages=false]{hyperref}
\fi
\definecolor{ChapterColor}{rgb}{0.510, 0.118, 0.000}
\definecolor{LinkColor}   {rgb}{0.741, 0.173, 0.000}
\definecolor{CiteColor}   {rgb}{0.000, 0.502, 0.118}
\definecolor{UrlColor}    {rgb}{0.000, 0.212, 0.800}
\hypersetup{linkcolor=LinkColor}
\hypersetup{citecolor=CiteColor}
\hypersetup{urlcolor=UrlColor}
%\hypersetup{pdflang={English}}
%%%%%%%%%%%%%%%%%%%%%%%%%%%%%%%%%%%%%%%%%%%%%%%%%%%%%%%%%%%%%%%%%%%%%%


%%% Local Variables: 
%%% mode: latex
%%% TeX-master: "Conventions"
%%% End: 

\ifpdf \usepackage{pdfsync} \fi
%%%%%%%%%%%%%%%%
%% Math forms %%
%%%%%%%%%%%%%%%%
\renewcommand{\d}{\ensuremath{\mathrm{d}}}
\newcommand{\e}{\ensuremath{\mathrm{e}}}
\renewcommand{\i}{\ensuremath{\mathrm{i}}}
\newcommand{\scri}{\ensuremath{\mathcal{I}}}
\newcommand{\Mirr}{\ensuremath{M_{\text{irr}}}}
\newcommand{\Eadm}{\ensuremath{M_{\text{ADM}}}}
\newcommand{\Jadm}{\ensuremath{J_{\text{ADM}}}}
\newcommand{\ADMMass}{\Eadm}
\newcommand{\IrrMass}{\Mirr}
\newcommand{\hyb}{\text{hyb}\xspace}
\newcommand{\NR}{\text{NR}\xspace}
\newcommand{\PN}{\text{PN}\xspace} \newcommand{\pN}{\PN}
\newcommand{\fc}{\ensuremath{f_{c}}}
\newcommand{\deff}{\ensuremath{D_{\text{eff}}}} % effective distance
\newcommand{\ta}{\ensuremath{t_{\text{a}}}}
\newcommand{\phia}{\ensuremath{\phi_{\text{a}}}}

%% Units
\renewcommand{\c}{\mathrm{c}}
\newcommand{\G}{\mathrm{G}}
\newcommand{\MSun}{\ensuremath{M_\odot}\xspace}
\newcommand{\Sun}{\MSun}

%% abs, norm, ceil, floor, avg
\newcommand{\abs} [1]{\left\lvert{#1}\right\rvert}
\newcommand{\norm}[1]{\left\lVert{#1}\right\rVert}
\newcommand{\ceil}[1]{\left\lceil{#1}\right\rceil}
\newcommand{\floor}[1]{\left\lfloor{#1}\right\rfloor}
\newcommand{\avg}[1]{\left\langle{#1}\right\rangle}

%% Frequencies and discretization
\newcommand{\fsamp}{\ensuremath{f_{\mathrm{s}}}}
\newcommand{\fNy}{\ensuremath{f_{\mathrm{Ny}}}}
\newcommand{\discretize}{\rightsquigarrow}
\DeclareMathOperator{\fft}{fft}
\DeclareMathOperator{\ifft}{ifft}


%%%%%%%%%%%%%%%%
%% Text forms %%
%%%%%%%%%%%%%%%%
\newcommand{\foreign}[1]{\textrm{#1}}
\newcommand{\etal}{\foreign{et~al.}\xspace}
\newcommand{\eg}{\foreign{e.g.}\xspace}
\newcommand{\ie}{\foreign{i.e.}\xspace}
\newcommand{\etc}{\foreign{etc.}\xspace}

\definecolor{NoteColor}{rgb}{0.900, 0.218, 0.000}
\newcommand{\Note}[1]{\textcolor{NoteColor}{[#1]}\index{Note}}
\definecolor{NewColor}{rgb}{0,.55,0}
\newcommand{\New}[1]{\textcolor{NewColor}{#1}\index{New}}

\newcommand{\software}[1]{\textsc{#1}}
\newcommand{\command}[1]{\texttt{#1}}


%%%%%%%%%%%%%%
%% Commands %%
%%%%%%%%%%%%%%
\newcommand{\CapName}[1]{\textbf{#1}.}

%% The following command allows you to enter \ShowDimensions anywhere
%% in a document, and -- at processing -- see all the dimensions (I
%% could think of) in the terminal.
\makeatletter
\newcommand{\ShowDimensions}{%
  \typeout{The font encoding is \f@encoding}        %
  \typeout{The font family is \f@family}            %
  \typeout{The font series is \f@series}            %
  \typeout{The font shape is \f@shape}              %
  \typeout{The font size is \f@size}                %
  \typeout{The baselineskip is \f@baselineskip}     %
  \typeout{The math font size is \tf@size}          %
  \typeout{The math script size is \sf@size}        %
  \typeout{The math scriptscript size is \ssf@size} %
  \typeout{The linewidth is \the\linewidth}         %
}
\makeatother


%%%%%%%%%%%%%%%%%%%%%%%%%%%%%%%%%%%%%%%%%%%%%%%%%%%%%%%%%%%%%%%%%%
%% My extension of the braket package for inner products        %%
%% This allows code like $\InnerProduct{s|h}$, which works much %%
%% like $\Braket{x|y}$.                                         %%
%%%%%%%%%%%%%%%%%%%%%%%%%%%%%%%%%%%%%%%%%%%%%%%%%%%%%%%%%%%%%%%%%%
\makeatletter
\let\protect\relax
{\catcode`\|=\active
  \xdef\InnerProduct{\protect\expandafter\noexpand\csname InnerProduct \endcsname}
  \expandafter\gdef\csname InnerProduct \endcsname#1{%
    \begingroup
    \ifx\SavedDoubleVert\relax
    \let\SavedDoubleVert\|\let\|\IpDoubleVert
    \fi
    \mathcode`\|32768\let|\IPVert
    \left({#1}\right)
    \endgroup
  }
}
\def\IPVert{\@ifnextchar|{\|\@gobble}% turn || into \|
     {\egroup\,\mid@vertical\,\bgroup}}
\def\IPDoubleVert{\egroup\,\mid@dblvertical\,\bgroup}
\let\SavedDoubleVert\relax
\def\midvert{\egroup\mid\bgroup}
\def\SetVert{\@ifnextchar|{\|\@gobble}% turn || into \|
    {\egroup\;\mid@vertical\;\bgroup}}
\def\SetDoubleVert{\egroup\;\mid@dblvertical\;\bgroup}
\def\mid@vertical{\mskip1mu\vrule\mskip1mu}
\def\mid@dblvertical{\mskip1mu\vrule\mskip2.5mu\vrule\mskip1mu}
\makeatother
\newcommand{\Overlap}{\Braket}


%%% Local Variables: 
%%% mode: latex
%%% TeX-master: "Conventions"
%%% End: 


\newcommand{\Caltech}{\affiliation{Theoretical Astrophysics 350-17,
    California Institute of Technology, Pasadena, CA 91125}}
\newcommand{\Cornell}{\affiliation{Center for Radiophysics and Space
    Research, Cornell University, Ithaca, New York, 14853}}


%%% Local Variables: 
%%% mode: latex
%%% TeX-master: "Conventions"
%%% End: 


%%%%%%%%%%%%%%%%

\begin{document}

%%%%%%%%%%%%%%%%


\graphicspath{{Plots/}}

\title{Conventions used in the Waveform code}

\author{Michael Boyle} \Cornell %

\date{\today}

\maketitle

%%%%%%%%%%%%%%%%%%%%%%%%%%%%%%%%%%%%%%%%%%%%%%%%%%%%%%%%%%%%%%%%%%%%%%%
%%%%%%%%%%%%%%%%%%%%%%%%%%%%%%%%%%%%%%%%%%%%%%%%%%%%%%%%%%%%%%%%%%%%%%%
\section{Basic Definitions}
\label{Sec:BasicDefinitions}

Regardless of the type of data ($\Psi_{4}$, $r\,M\,\Psi_{4}$, $h$,
\etc), it is assumed to be composed of a complex time series.  At each
time, the complex quantity is decomposed as
\begin{equation}
  z(t) \equiv A(t)\, \exp \left[ \i\, \phi(t) \right]\ ,
\end{equation}
where $A$ and $\phi$ are the data stored by the code.  Note especially
the sign in front of $\phi$.  (Also note that the amplitude may in the
future change to the $\log$ of the amplitude, so that the data stored
by the code may be defined by $z(t) \equiv \exp \left[ \chi(t) +
  \i\,\phi(t) \right]$.)  The phase data is immediately passed through
the \software{Matlab}/\software{Octave} function \command{unwrap},
which basically removes jumps of $2\,\pi$ in the phase.

When the data is or should be purely real (\eg, $m=0$), it is best to
redefine the data as $z(t) \equiv A(t)$ and $\phi(t)=0$.  In
particular, note that $A$ can be negative.  This is crucial for
extrapolation because the phases are typically contaminated by
numerical noise, and thus don't extrapolate well.


%%%%%%%%%%%%%%%%%%%%%%%%%%%%%%%%%%%%%%%%%%%%%%%%%%%%%%%%%%%%%%%%%%%%%%%
%%%%%%%%%%%%%%%%%%%%%%%%%%%%%%%%%%%%%%%%%%%%%%%%%%%%%%%%%%%%%%%%%%%%%%%
\section{Fourier transform conventions}
\label{sec:FourierTransformConventions}

The following conventions are chosen to be in agreement with
Matlab/Octave, as well as with the LSC.  The data returned by using
the function \command{Waveform/fft} is given by
Eq.~\eqref{eq:FourierConventionDiscrete}, while the frequencies of
those data are given by Eq.~\eqref{eq:DiscreteFrequencies}, with $j
\in [0, N-1]$.  The command \command{InnerProduct} returns the result
of the last expression in Eq.~\eqref{eq:InnerProductDiscretizedB} and
\command{Overlap} returns the result of
Eq.~\eqref{eq:OverlapDiscretizedC}.

We relate a continuous time signal $s(t)$ to its continuous Fourier
transform $\tilde{s}(f)$ by the formulas
\begin{align}
  \label{eq:FourierConventionContinuous}
  \tilde{s}(f) &\coloneqq \int_{-\infty}^{\infty}\, s(t)\, \e^{-2\pi
    \i f t}\, \d t\ ,
  \\
  \label{eq:FourierConventionContinuousInverse}
  s(t) &\equiv \int_{-\infty}^{\infty}\, \tilde{s}(f)\, \e^{2 \pi \i f
    t}\, \d f\ .
\end{align}
This normalization is chosen so that the transform of a pure signal
$s(t) = \e^{2\pi\i f t}$ will simply be a Dirac $\delta$ function,
with no additional factor.  Note that the entire range of $f$
(including negative frequencies) is necessary for general complex
functions.  However, if $s(t)$ is purely real, we have
\begin{equation}
  \label{eq:RealFourierTransformCondition}
  \tilde{s}(-f) = \left[\tilde{s}(f)\right]^{\ast}\ ,
\end{equation}
so that only the positive-frequency data are needed.

The notation for discretized signals involves subtleties.  Suppose the
continuous time signal $s(t)$ is sampled at $N$ uniform intervals,
beginning (for simplicity) at $t_{0}=0$ and separated by $\Delta t$.
This will give rise to a frequency discretization interval of $\Delta
f = 1/N\Delta t$.  Then, we define the quantities
\begin{gather}
  \label{eq:DiscreteFrequencies}
  f_{j} \coloneqq j \Delta f\ , \\
  \label{eq:DiscreteTimes}
  t_{k} \coloneqq k \Delta t\ ,
\end{gather}
for \emph{all} integers $j$ and $k$.  We will need to assume that the
time signal is periodic, over a time $N\Delta t$.  Under restriction
of the range of integration to one period and discretization
(represented by $\discretize$), the formulas for the Fourier transform
become
\begin{align}
  \label{eq:FourierConventionDiscretized}
  \tilde{s}(f_{j}) &\discretize
  \sum_{k=-\floor{(N-1)/2}}^{\floor{N/2}}\, s(t_{k})\, \e^{-2\pi\i
    f_{j} t_{k}}\, \Delta t \ \equiv\ \Delta t\, \sum_{k=0}^{N-1}\,
  s(t_{k})\, \e^{-2\pi\i jk/N}\ ,
  \\
  \label{eq:FourierConventionDiscretizedInverse}
  s(t_{k}) &\discretize \sum_{j=-\floor{(N-1)/2}}^{\floor{N/2}}\,
  \tilde{s}(f_{j})\, \e^{-2\pi\i f_{j} t_{k}}\, \Delta f \ \equiv\
  \Delta f\, \sum_{j=0}^{N-1}\, \tilde{s}(f_{j})\, \e^{-2\pi\i jk/N}\
  .
\end{align}
Note that we have used the assumption that $s(t+N\Delta t) = s(t)$ in
Eq.~\eqref{eq:FourierConventionDiscretized}.  We can also see from
this equation that $\tilde{s}(f_{j}) \leftrightsquigarrow
\tilde{s}(f_{j+N})$, which results from our restriction to a finite
range.  This fact is also used in
Eq.~\eqref{eq:FourierConventionDiscretizedInverse}.  We can transform
to the convention that $j,k \in [0, N-1]$ using the indexing set $(0,
\ldots, \floor{N/2}, -\floor{(N-1)/2}, \ldots, -1)$.  It is not hard
to see that the sequence thus defined is the same as $(0, \ldots,
N-1)$ modulo $N$.

Perhaps because $\tilde{s}(f_{j})$ as given by the expression
in~\eqref{eq:FourierConventionDiscretized} does not equal
$\tilde{s}(f)$ as given by Eq.~\eqref{eq:FourierConventionContinuous}
when $f=f_{j}$, there is a subtle notational distinction commonly
employed~\cite{Brown2004}.  We define $s_{k}\coloneqq s(t_{k})$, quite
naturally.  Using this, we write
\begin{align}
  \label{eq:FourierConventionDiscrete}
  \tilde{s}_{j} & \coloneqq \sum_{k=0}^{N-1}\, s_{k}\, \e^{-2 \pi \i j
    k / N}\ ,
  \\
  \label{eq:FourierConventionDiscreteInverse}
  s_{k} &\equiv \frac{1}{N}\, \sum_{j=0}^{N-1}\, \tilde{s}_{j}\, \e^{2
    \pi \i j k / N}\ .
\end{align}
Note that $\tilde{s}(f_{j}) \discretize \Delta t\, \tilde{s}_{j} =
\tilde{s}_{j} / N\, \Delta f$; in particular, $\tilde{s}(f_{j}) \neq
\tilde{s}_{j}$.  This normalization results in the circumstance that
the transform of a pure signal will be $\tilde{s}_{j} = N$ for the
relevant frequency and $\tilde{s}_{j} = 0$ for $j \in [0,
N-1]$---which is roughly the discrete version of the $\delta$
function.  Of course, this means that $\tilde{s}_{j}$ depends on the
sampling frequency; for a given signal, $\tilde{s}_{j}$ will typically
scale as $N$.  This notation and choice of normalization agree with
the standard conventions of LIGO~\cite{AndersonEtAl2001} and the
\software{FFTW} and \software{Matlab}/\software{Octave} software
packages.  In particular, \software{Matlab}/\software{Octave} returns
$\tilde{s}_{j}$ as the result of the command \command{fft}.  As a
result, Waveform objects returned by \command{Waveform/fft} also use
this convention.

The assumption that the time-domain signal is periodic, and the
restriction of the Fourier
transform---Eq.~\eqref{eq:FourierConventionContinuous}---to a finite
range make the frequency-domain signal periodic on the scale of the
sampling frequency
\begin{equation}
  \label{eq:SamplingFrequency}
  \fsamp \coloneqq \frac{1}{\Delta t}\ .
\end{equation}
Any frequency $f \geq \fsamp$ will be \emph{aliased} to a lower
frequency, meaning that power will appear at a frequency $f-\fsamp$.
In fact, any continuous signal at frequency $f$ will be represented in
the discretized case by a Fourier transform with nonzero values at
$f+n\fsamp$, for \emph{all} integers $n$.  That is, what was a
$\delta$ function in the continuous case has been made into a
frequency ``comb'' (with infinitely many finitely large teeth) in the
discrete case.

More stringent than the cutoff of the sampling frequency, there is an
upper limit to the frequencies that can be unambiguously represented
on a given discrete grid: the Nyquist frequency
\begin{equation}
  \label{eq:NyquistFrequency}
  \fNy \coloneqq \frac{1}{2\Delta t} = \frac{\fsamp}{2}\ .
\end{equation}
The ``comb'' effect combines with the negative-frequency effect to
produce this limit.  [Note that $\sin(2 \pi \fNy t)$ cannot be
represented on the discrete set of points because it is zero at each
$t_{k}$, but $\cos(2 \pi \fNy t)$ can be.]  Typically, we restrict
attention to frequencies $f \in (-\fNy, \fNy]$ or $f \in [0, 2\fNy)$,
so that only one ``tooth'' is present in the data for each component
of a signal.  When $f \in [0, 2\fNy)$, we make use of the fact that
$\tilde{s}_{k} = \tilde{s}_{k \bmod N}$ to wrap negative frequencies
onto frequencies greater than $\fNy$.  In particular, note that
$\tilde{s}_{-\floor{(N-1)/2}} = \tilde{s}_{\floor{(N+1)/2}}$ and
$\tilde{s}_{-1} = \tilde{s}_{N-1}$.


\subsection{Inner products and overlaps}
\label{sec:InnerProductsAndOverlaps}

Current searches for gravitational waves from binary black hole
coalescence use matched filtering to search for a waveform buried in
noise.  The matched filter is the optimal filter for detecting a
signal in stationary Gaussian noise.  Suppose that $n(t)$ is a
stationary Gaussian noise process with one-sided power spectral
density $S_n(f)$ given by $\langle \tilde{n}(f) \tilde{n}^\ast(f')
\rangle=\frac{1}{2} S_n(|f|)\delta(f-f')$.  For long integration
times, the data stream $s(t)$ output by the detector will always be
dominated by the noise.  Thus, we can simply approximate $n \approx s$
to calculate $S_{n}(f)$.

Using this power spectral density (PSD), we can define the inner
product between two real-valued signals---the data stream $s$ and the
filter template $h$---by
\begin{subequations}
  \label{eq:InnerProduct}
  \begin{align}
    \InnerProduct{s|h} &\coloneqq 2\, \Re \int_{-\infty}^{\infty}\,
    \frac{\tilde{s}(f)\, \tilde{h}^{\ast}(f)}{S_{n}(\abs{f})}\, \d f
    \\ &= 4\, \Re \int_{0}^{\infty}\, \frac{\tilde{s}(f)\,
      \tilde{h}^{\ast}(f)}{S_{n}(f)}\, \d f\ .
  \end{align}
\end{subequations}
For simplicity of presentation, we will assume that the signals are
normalized, so that $\InnerProduct{s|s}=1$ and $\InnerProduct{h|h}=1$.
Obviously, this condition can always be imposed, if necessary, by
taking
\begin{equation}
  \label{eq:NormalizedWaveforms}
  s \to \frac{s}{\sqrt{\InnerProduct{s|s}}} \quad \text{and} \quad h
  \to \frac{h}{\sqrt{\InnerProduct{h|h}}}\ .
\end{equation}

Discretization of the above equation is not entirely straightforward.
In real detectors (\eg, LIGO), the data has a sampling frequency---and
an associated Nyquist frequency---where the assumption is that no
signal of interest occurs at higher frequencies.  Thus, we assume that
the Fourier transforms are zero for $f \notin (-\fNy, \fNy]$.  Using
this fact, we can limit the range of the integrals and discretize, to
find
\begin{subequations}
  \label{eq:InnerProductDiscretized}
  \begin{align}
    \label{eq:InnerProductDiscretizedA}
    \InnerProduct{s|h} &\discretize 2\, \Delta f\, \Re\,
    \sum_{j=-\floor{(N-1)/2}}^{\floor{N/2}}\, \frac{\tilde{s}(f_{j})\,
      \tilde{h}^{\ast}(f_{j})}{S_{n}(\abs{f_{j}})} = \frac{2}{N^{2}
      \Delta f}\, \Re\, \sum_{j=-\floor{(N-1)/2}}^{\floor{N/2}}\,
    \frac{\tilde{s}_{j}\, \tilde{h}^{\ast}_{j}}{S_{n}(\abs{f_{j}})} =
    \frac{2}{N^{2} \Delta f}\, \Re\, \sum_{j=0}^{N-1}\,
    \frac{\tilde{s}_{j}\, \tilde{h}^{\ast}_{j}}{S_{n}(\abs{f_{j}})}
    \\
    \label{eq:InnerProductDiscretizedB}
    &\discretize 4\, \Delta f\, \Re \sum_{j=0}^{\floor{N/2}}\,
    \frac{\tilde{s}(f_{j})\, \tilde{h}^{\ast}(f_{j})} {S_{n}(f_{j})} =
    \frac{4}{N^{2} \Delta f}\, \Re \sum_{j=0}^{\floor{N/2}}\,
    \frac{\tilde{s}_{j}\, \tilde{h}^{\ast}_{j}} {S_{n}(f_{j})}\ .
  \end{align}
\end{subequations}
Note the apparent discrepancy here between
Eqs.~\eqref{eq:InnerProductDiscretizedA}
and~\eqref{eq:InnerProductDiscretizedB}: the $j=0$ term is counted
twice in Eq.~\eqref{eq:InnerProductDiscretizedB} as compared to the
previous one, as is the $N/2$ term when $N$ is even.  Of course, $j=0$
corresponds to the $f=0$ part of the signal; for this value, the noise
is effectively infinite, and no physical signal is expected to be
discernible, so such a term should be ignored in any case.  Similarly,
the noise is quite large (though certainly finite) and the signal
small when $j=N/2$.  Moreover, $\fNy$ is typically chosen in the hope
that there is no signal of interest at that frequency, so
$\tilde{s}_{N/2}$ should be effectively zero.  Thus, while there is a
formal difference between the two versions of the discretization,
there should be no practical difference.

The filter template includes arbitrary time and phase offsets, encoded
by the arrival time and phase, $\ta$ and $\phia$.  Under a change of
these quantities, the Fourier transform behaves as
\begin{equation}
  \label{eq:EffectOfTimeAndPhaseOffset}
  \tilde{h}(f) \to \tilde{h}(f)\, \e^{-2\pi\i f \ta - \i \phia}\ .
\end{equation}
The matched-filter output---the \emph{overlap} between the two
waveforms---is then defined as the (normalized) inner product of the
signals, maximized over these two variables:
\begin{subequations}
  \begin{align}
    \label{eq:OverlapDefinition}
    \Overlap{s|h} &\coloneqq \max_{\ta, \phia}\, \InnerProduct{s|h} \\
    &= \max_{\ta, \phia}\, 2\, \Re \int_{-\infty}^{\infty}\,
    \frac{\tilde{s}(f)\, \tilde{h}^{\ast}(f)}{S_{n}(f)}\, \e^{2\pi \i
      f\ta + \i \phia}\, \d f
    \\
    & = 2 \max_{\ta}\, \abs{\int_{-\infty}^{\infty}\,
      \frac{\tilde{s}(f)\, \tilde{h}^{\ast}(f)}{S_{n}(f)}\, \e^{2\pi
        \i f\ta}\, \d f }\ .
  \end{align}
\end{subequations}
This integral is just the (inverse) Fourier transform of the quantity
$\tilde{s}(f)\, \tilde{h}^{\ast}(f) / S_{n}(f)$ evaluated at $\ta$,
which allows for fast computation, using FFTs.  In the discrete case,
finding the maximum over $\ta$ involves taking the Fourier transform
and selecting the largest element of the finite set that results from
discretization:
\begin{subequations}
  \label{eq:OverlapDiscretized}
  \begin{align}
    \label{eq:OverlapDiscretizedA}
    \Overlap{s|h} &\discretize 2 \Delta f\, \max_{\ta}\,
    \abs{\sum_{j=-\floor{(N-1)/2}}^{\floor{N/2}}\,
      \frac{\tilde{s}(f_{j})\,
        \tilde{h}^{\ast}(f_{j})}{S_{n}(\abs{f_{j}})}\, \e^{2\pi \i
        f_{j} \ta}}
    \\
    \label{eq:OverlapDiscretizedB}
    &= \frac{2}{N \Delta f}\, \max_{k}\, \abs{\frac{1}{N}\,
      \sum_{j=0}^{N-1}\, \frac{\tilde{s}_{j}\,
        \tilde{h}^{\ast}_{j}}{S_{n}(\abs{f_{j}})}\, \e^{2\pi \i j k /
        N}}
    \\
    \label{eq:OverlapDiscretizedC}
    &= \frac{2}{N \Delta f}\, \max\, \abs{\ifft \left[
        \frac{\tilde{s}_{j}\,
          \tilde{h}^{\ast}_{j}}{S_{n}(\abs{f_{j}})} \right]}\ .
  \end{align}
\end{subequations}
Note, however, that the maximum may occur for any of the $N$ values of
time.  Therefore, the full transform---including all $N$
frequencies---must be performed.

For \emph{non-normalized} real data $s(t)$ and a \emph{normalized}
optimal template $h(t)$, the signal-to-noise ratio (SNR) is defined as
\begin{equation}
  \label{eq:SNR}
  \rho_{s,\, h} \coloneqq \Overlap{s|h}\ .
\end{equation}
Because we assume $h$ approximates $s$ very well, we can approximate
the SNR by simply substituting the normalized $s$ for $h$:
\begin{equation}
  \label{eq:ApproxSNR}
  \rho_{s,\,h}\approx\frac{\Overlap{s|s}}{\sqrt{\InnerProduct{s|s}}} =
  \sqrt{\InnerProduct{s|s}}\ .
\end{equation}
Note that the scaling of $s$ is inverse with respect to the distance
to the source, $R$.  Thus, the SNR of the signal varies inversely as
the distance.  If we set a threshold SNR, then, the distance to which
the signal can be ``seen'' (exceeds that threshold) is inversely
proportional to the SNR.  Because the event rate is roughly
proportional to the volume, $R^{3}$, ``seen'' by the detector, the
event rate is roughly proportional to the SNR.  Thus, if we improve
the SNR of a given signal (\eg, by decreasing the noise) by a factor
of 2, the event rate goes up by roughly $2^{3}=8$.


\subsection{Power Spectral Densities (PSDs)}
\label{sec:PowerSpectralDensities}

bla



%%%%%%%%%%%%%%%%%%%%%%%%%
\bibliography{References}
%%%%%%%%%%%%%%%%%%%%%%%%%


%%%%%%%%%%%%%%%%
\end{document}
%%%%%%%%%%%%%%%%


%%% Local Variables: 
%%% mode: latex
%%% TeX-master: t
%%% End: 
